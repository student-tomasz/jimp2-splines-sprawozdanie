\documentclass[10pt,a4paper]{article}
\usepackage[a4paper]{geometry}

\usepackage{polski}
\usepackage{xltxtra}

\usepackage{fancyvrb}
\usepackage{relsize}
\usepackage{alltt}
\usepackage{amsfonts}
\usepackage[pdfborder={0 0 0}]{hyperref}

%% tweak fonts
\defaultfontfeatures{Mapping=tex-text}
\setromanfont{Charis SIL}
%\setsansfont[Scale=MatchLowercase]{Gill Sans}
\setmonofont[Scale=MatchLowercase]{Menlo}
\linespread{1.25}

%% define custom commands and environments
\DefineVerbatimEnvironment%
  {SmallVerbatim}%
  {Verbatim}{fontsize=\relsize{-0.5},numbers=left,numbersep=-10pt,frame=lines,tabsize=4}
\newcommand{\p}[1]{\texttt{#1}}
\newcommand{\flag}[1]{\textbf{\p{#1}}}

\begin{document}

%%fakesection{Tytuł}
\title{ 
  Interpolacja funkcjami sklejanymi\\
  {\normalsize Sprawozdanie końcowe z~projektu nr 1}\\\vspace{-12pt}
  {\normalsize z przedmiotu \emph{Języki i metody programowania 2}}
}
\author{
  Tomasz Cudziło\\
  {\small EE PW, 211552}
}
\date{\today}
\maketitle

\section*{Zadanie}
\label{sec:zadanie}

Napisać program wyznaczający współczynniki funkcji sklejanych trzeciego stopnia
aproksymujących zadany ciąg danych pomiarowych.

\vspace{24pt}

\section{Zmiany w~funkcjonalności}

\section{Zmiany w~implementacji}

\section{Ograniczenia programu}

\section{Testy}

\end{document}
