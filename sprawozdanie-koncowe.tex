\documentclass[10pt,a4paper]{article}
\usepackage[a4paper]{geometry}

\usepackage{polski}
\usepackage{xltxtra}

\usepackage{fancyvrb}
\usepackage{relsize}
\usepackage{alltt}
\usepackage{amsfonts}
\usepackage[pdfborder={0 0 0}]{hyperref}

\usepackage{xr}
\externaldocument[func-]{../func/specyfikacja-funkcjonalna}
\externaldocument[impl-]{../impl/specyfikacja-implementacyjna}

%% tweak fonts
\defaultfontfeatures{Mapping=tex-text}
\setromanfont{Charis SIL}
%\setsansfont[Scale=MatchLowercase]{Gill Sans}
\setmonofont[Scale=MatchLowercase]{Menlo}
\linespread{1.25}

%% define custom commands and environments
\DefineVerbatimEnvironment%
  {SmallVerbatim}%
  {Verbatim}{fontsize=\relsize{-0.5},numbers=left,numbersep=-10pt,frame=lines,tabsize=4}
\newcommand{\p}[1]{\texttt{#1}}
\newcommand{\flag}[1]{\textbf{\p{#1}}}

\begin{document}

%%fakesection{Tytuł}
\title{ 
  Interpolacja funkcjami sklejanymi\\
  {\normalsize Sprawozdanie końcowe z~projektu nr 1}\\\vspace{-12pt}
  {\normalsize z przedmiotu \emph{Języki i metody programowania 2}}
}
\author{
  Tomasz Cudziło\\
  {\small EE PW, 211552}
}
\date{\today}
\maketitle

\section*{Zadanie}
\label{sec:zadanie}

Napisać program wyznaczający współczynniki funkcji sklejanych trzeciego stopnia
aproksymujących zadany ciąg danych pomiarowych.

\vspace{24pt}

\section{Zmiany w~funkcjonalności}
Możliwości programu nie odbiegają od funkcjonalności opisanej w~specyfikacji
funkcjonalnej projektu, oprócz formatu pliku tworzonego dla {\tt gnuplot}.

\subsection{Plik dla \texttt{gnuplot}}
Zmienił się szablon pliku tworzonych przez program opisany
w~rozdziale~\ref{func-sec:plik_gnuplot} specyfikacji funkcjonalnej:
\begin{SmallVerbatim}
    # created at: Wed Apr  6 19:56:50 2011
    #       file: correct.plt
    set term pdf
    set output "correct.pdf"
    set nokey
    set notitle
    set grid

    set parametric

    x0 = 1; y0 = 2;
    x1 = 3; y1 = 3.5;
    x2 = 5; y2 = 3.7;

    x0(t) = x0+(x1-x0)*t;
    x1(t) = x1+(x2-x1)*t;
    x2(t) = x2+(x3-x2)*t;

    fun0(x) =   1.128125 +   0.790625*x +   0.121875*x**2 +  -0.040625*x**3
    fun1(x) =  -1.065625 +   2.984375*x +  -0.609375*x**2 +   0.040625*x**3

    plot [t=0:1]\
      x0(t), fun0(x0(t)),\
      x1(t), fun1(x1(t)),\
      "-" with points
        1 2
        3 3.5
        5 3.7
      e
\end{SmallVerbatim}

\section{Zmiany w~implementacji}

\section{Ograniczenia programu}

\section{Testy}

\end{document}
